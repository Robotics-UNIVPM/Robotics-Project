\documentclass{article}
\usepackage[utf8]{inputenc} %per i caratteri accentati
\usepackage{mathtools}

\begin{document}

\noindent
Siano $l(t)$, $r(t)$ le distanze registrate fino al tempo $t$ sulle
ruote motrici sinistra e destra rispettivamente (left e right). Se $L$ è la
distanza tra le ruote, l'imbardata, in radianti, cioè l'angolo tra l'asse x e
la direzione in cui punta il robot, è

\begin{equation}
	\theta (t) = \theta_0 + \frac{r(t)-l(t)}{L},
\end{equation}

\noindent
quindi il versore della traiettoria è

\begin{equation}
	\hat{v} (t) = \begin{bmatrix} cos\theta (t) \\ sin\theta (t) \end{bmatrix}.
\end{equation}

\noindent
Inoltre la velocità è la media delle velocità registrate

\begin{equation}
	v(t)=||v(t)||\hat{v}(t) = \frac{r'(t)+l'(t)}{2}\begin{bmatrix}cos\theta (t) \\
	sin\theta (t) \end{bmatrix}.
\end{equation}

\noindent 
Da cui la traiettoria:

\begin{equation}
	\begin{bmatrix}x(t)\\ y(t) \end{bmatrix} = \begin{bmatrix}x_0 \\ y_0
	\end{bmatrix} +\int_0^t v(\tau )d\tau = \begin{bmatrix}x_0 + \int_0^t
	\frac{r'(t)+l'(t)}{2} cos(\theta_0 + \frac{r(t)-l(t)}{L})d\tau \\ y_0 +
	\int_0^t \frac{r'(t)+l'(t)}{2} sin(\theta_0 + \frac{r(t)-l(t)}{L})d\tau
	\end{bmatrix}.
\end{equation}

\end{document}
